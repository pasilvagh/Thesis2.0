\chapter{Glosario}
\label{Glosario}

Para empezar este estudio es necesario introducir ciertas nociones y lenguaje que se usarán durante todo el documento. Estos conceptos son usados en la seguridad y Desarrollo de Software, y son extendibles para lo que se verá en este estudio.

\begin{itemize}
    \item Seguridad - \textit{Security}:
        \\Es una Propiedad que podría tener un sistema, donde asegura la protección de los recursos e información, en contra de ataques maliciosos desde fuentes externas como internas. La seguridad también involucra controlar que el funcionamiento de un sistema sea como debería ser, y que nada externo o interno genere un error.
    \item Error - \textit{Error}:
        \\Es una acción de caracter humano. Éste se genera cuando se tienen ciertas nociones equivocadas, que causan un Defecto en el Sistema o Código.
    \item Defecto  - \textit{Defect}:
        \\Es una caracterítica que se obtiene a nivel de Diseño, cuando una funcionalidad no hace lo que tiene que realmente hacer. Según la IEEE CSD o \textit{Center for Secure Design} \cite{ieeecsd2}, un defecto puede ser subdividido en 2 partes: falla o \textbf{flaw} y \textbf{bug}, donde la primera tiene que ver con un error de \textbf{alto nivel}, mientras que un bug es un problema de implementación en el Software. Una falla es menos notoria que un bug, dado que ésta es de caracter abstracto, a nivel de diseño del Software.
    \item Falla - \textit{Fail/Flaw}:
        \\Es un estado en que el Software Implementado no funciona como debería de ser.
    \item Vulnerabilidades - \textit{Vulnerability}:
        \\Es una debilidad inherente del sistema que permite a un atacante poder reducir el nivel de confianza de la información de un sistema. Una vulnerabilidad convina 3 elementos: un \textbf{defecto} en el sistema, un \textbf{atacante} tratando de acceder a ese defecto y la \textbf{capacidad} que tiene el atacante para llevarlo a cabo. Particularmente las vulnerabilidades más críticas son documentadas en la \textit{Common Vulnerabilities and Exposures} (CVE) \cite{cve}.
    \item Superficie de Ataque - \textit{Attack Surface}:
        \\Es el conjunto de todas las posibles vulnerabilidades que un sistema puede tener, en un cierto momento, para una cierta versión del sistema, etc.
    \item Amenaza - \textit{Threat}
        \\Es una acción/evento realizada por un sujeto que se aprovecha de las vulnerabilidades del sistema, debilidades, para causar un daño, y que dependiendo del recurso al que afecte el daño puede o no ser reparable.
    \item Ataque - \textit{Attack}
        \\Es el éxito de la amenaza en el aprovechamiento de la vulnerabilidad (explotación de ésta), de tal forma que genera una acción negativa en el sistema y favorable para el atacante.
    \item \textit{Exploit}:
        \\Usar una pieza de software para poder llevar a cabo un ataque sobre un objetivo, intentando \textbf{explotar} la vulnerabilidad de éste. Este tipo de acción permite en consecuencia obtener control en el sistema computacional, en donde la vulnerabilidad permitió su acceso.
    \item Ingeniería Social - \textit{Social Engineering}
        \\El acto de manipular a las personas de manera que realicen acciones o divulguen información confidencial. El término aplica al acto de engañar con el propósito de juntar información, realizar un fraude, u obtener acceso a un sistema computacional. La definición anterior encontrada en Wikipedia es extendida por el autor del libro ``The Social Engineer's Playbook'' \cite{socEngineeering}, donde agrega que además la Ingeniería Social involucra el hecho de manipular a una persona en realizar acciones que finalmente no son para beneficiar a la víctima. Un ataque de éste tipo también puede llegar a ser realizado tanto \textbf{cara a cara}, como de forma indirecta. Pero el autor del libro indica que siempre hay un \textbf{contacto} previo con la víctima.
    \item Confidencialidad - \textit{Confidentiality} 
        \\Característica o propiedad que debe mantener un sistema para que la información privilegiada de alguna entidad que depende de tal sistema, no sea develada a nadie más que al que le pertenece la información.
    \item Integridad - \textit{Integrity}
        \\Característica o propiedad que asegura que la información no será modificada/alterada nada más que por la entidad a quién le pertenece y con el previo consentimiento de éste.
    \item Disponibilidad - \textit{Availability}
        \\Característica o propiedad que permite que la información esté disponible para quién lo necesite, en el momento que sea. La imposibilidad de obtener datos en un cierto instante de tiempo, conlleva a la pérdida de esta propiedad.
    \item \textit{Phishing}
        \\Técnica de Ingeniería Social. Mediante el uso de correo elentrónico, links (url's), acortamiento de urls y otras herramientas, se busca que una víctima visite un sitio o aprete un link de manera que se de la \textbf{autorización explícita} del usuario para descargar código malicioso o enviar datos a un servidor malicioso. El objetivo de esta técnica es poder obtener información valiosa de la victima o relizar algún daño en el cliente web.
    \item \textit{Malware}
        \\Software creado para realizar acciones maliciosas en los datos o sistema de un usuario. La mayoría de las veces puede ser instalado con el permiso del usuario realizando una técnica de Phishing, y otras veces engañando al usuario a que acepte sin darse cuenta.
    \item \textit{Man-in-the-Middle}
        \\Ataque que causa una pérdida en la Confidencialidad o la Integridad de la información que es revelada. La causa de este ataque puede ser por ataques de tipo Phishing, como a través de vulnerabilidades del sistema que debieron ser explotadas antes para causar el ataque MiTM.
    \item \textit{Fingerprinting}
        \\Es la acción de recolectar información de un dispositivo o sistema remoto para poder identificar a quién esté detrás de él.
    \item \textit{Timing attack}
        \\Es un tipo de \textit{side channel attack}, en donde el atacante intenta comprometer la implementación criptográfica de un sistema al analizar el tiempo que le toma ejecutar el algoritmo criptográfico.
    %\item \textit{Penetration Testing}
    %\item \textit{Fuzzing}
\end{itemize}
